\documentclass{report}
\usepackage{tikz}
\usepackage{xcolor}
\input{preamble}
\input{macros}
\input{letterfonts}

\title{\Huge{AP Calculus BC}\\}
\author{\Huge{Xin D.}}
\date{\huge{12.20.23}}

\begin{document}

\maketitle

\qs{}{

\begin{align*}
	\int_3^e\left(\frac{x^3 - x}{x^2}\right)dx
\end{align*}

}

\pf{Proof}{
	\begin{align*}
		\int_3^e\left(\frac{x^3}{x^2} - \frac{x}{x^2}\right)dx \\\\ 
		\text{Cancel the terms out:}
	\end{align*}
	\begin{align*} 
		\int_3^e\left(x - \frac{1}{x}\right) \\\\
		\text{Integrate:}
	\end{align*}
	\begin{align*}
		= \Bigg[\frac{x^2}{2} - \ln|x|\Bigg]_3^e 
	\end{align*}
	\begin{align*}
		\text{Plug in for the definite integral: F(b) - F(a):}
	\end{align*}
	\begin{align*}
		\left(\frac{e^2}{2} - lne\right) - \left(\frac{9}{2} - ln3\right)
	\end{align*}
	\begin{align*}
		\text{Remember that} \ln \text{e = 1}
	\end{align*}
	\begin{align*}
		&= \left(\frac{e^2}{2} - 1\right) - \frac{9}{2} + \ln3 \\\\
		&= \frac{e^2}{2} - \frac{2}{2} - \frac{9}{2} + \ln3 \\\\
		&= \colorbox{yellow}{$\displaystyle \frac{e^2}{2} - \frac{11}{2} + \ln3$} \\\\\\\\\\
	\end{align*}
}
	

\qs{}{
	\begin{equation}
		\int \tan^2{2x } dx
	\end{equation}
}

\pf{Proof}{
	\begin{align*}
		\text{Remember that the } \int \tan^2{2x} = \it{1 - \sec^2{2x}} 
	\end{align*}

	\begin{align*}
		\text{Rewrite:}
	\end{align*}

	\begin{align*}
		\int \left(1-\sec^2{2x}\right) dx 
	\end{align*}

	\begin{align*}
		\text{Let u = 2x} \\\\
		\text{du = 2dx} \\\\
		\frac{du}{2} = dx
	\end{align*}

	\begin{align*}
		\int\left(1 - \sec^2(2x)\right) \frac{du}{2}
	\end{align*}
	\begin{align*}
		\text{Separate into two integrals:}
	\end{align*}
	\begin{align*}
		\int \Big[1\Big] - \int \Big[-\sec^2(2x)\Big] \frac{du}{2} \\ 
	\end{align*}
	\begin{align*}
		\text{Remember that } \int \sec^2{u}du = \tan u + C 
	\end{align*}
	\begin{align*}
		\text{Evaluate:}
	\end{align*}
	\begin{align*}
		x + \big[- \tan{2x} + C\big] \frac{du}{2} 
	\end{align*}
	\begin{align*}
		\text{Note that the constant 1/2 is only applied to the tan}
	\end{align*}
	\begin{align*}
		\colorbox{yellow}{$\displaystyle x - \frac{1}{2} \tan{2x} + C$}
	\end{align*}
}


REVIEW OF BASIC
INTEGRATION RULES $(a>0)$\\\\
1. $\int k f(u) d u=k \int f(u) d u$\\\\
2. $\int[f(u) \pm g(u)] d u= \int f(u) d u \pm \int g(u) d u$ \\\\
3. $\int d u=u+C$ \\\\
4. $\int u^n d u=\frac{u^{n+1}}{n+1}+C, n \neq-1$ \\\\
5. $\int \frac{d u}{u}=\ln |u|+C$ \\\\
6. $\int e^u d u=e^u+C$ \\\\
7. $\int a^{\prime \prime} d u=\left(\frac{1}{\ln a}\right) a^u+C$ \\\\
8. $\int \sin u d u=-\cos u+C$ \\\\
9. $\int \cos u d u=\sin u+C$ \\ \\
10. $\int \tan u d u=-\ln |\cos u|+C$ \\\\
11. \colorbox{yellow}{$\displaystyle \int e^{ax} \, dx = \frac{1}{a}e^{ax}$} \\\\\\


PROCEDURES FOR FITTING INTEGRANDS TO BASIC INTEGRATION RULE \\\\

Expand (numerator).
\begin{align*}
\left(1+e^x\right)^2=1+2 e^x+e^{2 x} \\
\end{align*}

Separate numerator.
\begin{align*}
	\frac{1+x}{x^2+1}=\frac{1}{x^2+1}+\frac{x}{x^2+1} \\
\end{align*}

Complete the square.
\begin{align*}
	\frac{1}{\sqrt{2 x-x^2}}=\frac{1}{\sqrt{1-(x-1)^2}} \\
\end{align*}

Divide improper rational function.
\begin{align*}
	\frac{x^2}{x^2+1}=1-\frac{1}{x^2+1} \\
\end{align*}

Add and subtract terms in numerator.
\begin{align*}
	\frac{2 x}{x^2+2 x+1}=\frac{2 x+2-2}{x^2+2 x+1}=\frac{2 x+2}{x^2+2 x+1}-\frac{2}{(x+1)^2} \\
\end{align*}

Use trigonometric identities.
\begin{align*}
	\cot ^2 x=\csc ^2 x-1 \\
\end{align*}

Multiply and divide by Pythagorean conjugate.





\begin{align*}
	\frac{1}{1+\sin x}=\left(\frac{1}{1+\sin x}\right)\left(\frac{1-\sin x}{1-\sin x}\right)=\frac{1-\sin x}{1-\sin ^2 x}
\end{align*}
\begin{align*}
	=\frac{1-\sin x}{\cos ^2 x}=\sec ^2 x-\frac{\sin x}{\cos ^2 x} \\
\end{align*}

\qs{}{
	\begin{align*}
		\text{Evaluate:}
	\end{align*}
	\begin{equation}
		\int_0^1\frac{x+3}{\sqrt[2]{4-x^2}} dx
	\end{equation}
}
\pf{Proof}{
	\begin{align*}
		\text{Separate into two terms for integration:}
	\end{align*}
	\begin{align*}
		\int_0^1\left(\frac{x}{\sqrt[]{4-x^2}} + \frac{3}{\sqrt[]{4-x^2}}\right) dx
	\end{align*}
}
\begin{align*}
	\LARGE\text{NOTES \:FOR\: TEST}  \\
\end{align*}
\begin{equation*}
	\colorbox{yellow}{$\displaystyle \int e^{ax} \, dx = \frac{1}{a}e^{ax}$} \\\\
\end{equation*}
\begin{align*}
	\text{Total distance = } \int_a^b \big| v(t) \big|
\end{align*}
\begin{align*}
	\int\frac{1}{e}dt = \frac{1}{e}t + C,\: \text{note that}\;\frac{1}{e}\;\text{is a fraction}
\end{align*}
\begin{align*}
	\text{Average Value Formula: }\; \frac{1}{b-a} \int_{a}^{b}f(x)\,dx
\end{align*}
\\\\\\\\
\\\\
\\\\


$\large\text{Example:}$ 
\begin{align*}
	\text{Find the average value of the function} \: f(x) = \frac{2}{1+x} \; \text{from x = 1 to x = 7}
\end{align*}

$\;\;\;\;\;\;\;\;\;\;\;\;\;\;\;\;\;\;\;\;\;\text{Express your answer as a constant times ln2.}$
\\ \pf{Proof}{
	\begin{align*}
		\frac{1}{7-1}\:\int_{1}^{7}\:\frac{2}{1+x}\;dx \\\\
	\end{align*}
	\text{Take the anti-derivative, take constant out:} 
	\begin{align*}
		\frac{1}{6} * 2\ln\big|1+x\big|\;\Bigg|_1^7
	\end{align*}
	\text{Evaluate and plug in:}
	\begin{align*}
		\frac{1}{3} \ln\left(\frac{8}{2}\right) = \frac{1}{3}\ln\left(4\right)
	\end{align*}
	\text{Rewrite the 4:}
	\begin{align*}
		\frac{1}{3} \ln\left(2^2\right)
	\end{align*}
	\text{Take the power out according to Logarithic Rules:}
	\begin{align*}
		\frac{1}{3}*2*\ln(2)=\frac{2}{3} \ln\left(2\right)
	\end{align*}
	\begin{align*}
		\colorbox{yellow}{$\displaystyle \frac{2}{3} \ln\left(2\right)$}
	\end{align*}
	\\\\\\\\\\\\\\\\\\\\\\\\\\\\\\\\\\\\
}


\begin{align*}
	\Large\text{Trig with Motion, Solving for primary zeroes:}
\end{align*} \\\\
\large{A particle moves along the x-axis with position given by $\bold{x(t) = 2\sin\left(\frac{\pi}{5}\,t\right) + 3}$\\\\ Find all times in the interval $0 \leq t < 10$}\\
\pf{Proof}{
	\begin{align*}
		\text{Find velocity function first, take the derivative of position function:}                                                                                     
	\end{align*}
	\begin{align*} 
		v(t) = 2\cos\left(\frac{\pi}{5}t\right) * {\frac{\pi}{5}}
	\end{align*}
	\begin{align*}
		\text{Set the velocity function to zero and solve for t:}
	\end{align*}
	\begin{align*}
		0 = \frac{2\pi}{5}\cos\frac{\pi}{5}t
	\end{align*}
	\begin{align*}
		\text{Divide the constant out:}
	\end{align*}
	\begin{align*}
		0 = \cos \frac{\pi}{5}t
	\end{align*}
	\begin{align*}
		\text{Remember that $\cos$ is only equal to 0 at the muliples of $\pi$, starting at $\frac{\pi}{2}$, e.g.: $\frac{\pi}{2}$, $\frac{3\pi}{2}$, $\frac{5\pi}{2}$... }
	\end{align*}
	\begin{align*}
		\text{We need to make the term inside the $\cos$ to be equal to one of those values} \\
	\end{align*}
	\begin{center}
		\begin{tabular}{ c c }
		  $\frac{\pi}{5} t = \frac{\pi}{2}$ & $\frac{\pi}{5} t = \frac{3\pi}{2}$ \\    
		\end{tabular}
	\end{center}
	\begin{align*}
		\text{Divide:}
	\end{align*}

	\begin{center}
		\begin{tabular}{c c}
			$t = \frac{5}{2}$ & $t = \frac{15}{2}$ 
		\end{tabular}
	\end{center}

	\begin{align*}
		\Large{\colorbox{yellow}{\text{$\left(\frac{5}{2},\frac{15}{2}\right)$}}}
	\end{align*}\\\\\\\\\\\\\\\\\\\\\\
} 

\begin{align*}
	\Large{\colorbox{yellow}{\text{$\frac{d}{dx}\,a^{bx} = a^{bx} * b * \ln a$}}}
\end{align*}
\begin{align*}
	\Large{\colorbox{yellow}{\text{$\int\,a^{bx} = \frac{a^{bx}}{b\,\ln a}$}}}
\end{align*}
	





\end{document}
